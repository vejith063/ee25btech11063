\let\negmedspace\undefined
\let\negthickspace\undefined
\documentclass[journal]{IEEEtran}
\usepackage[a4paper, margin=10mm, onecolumn]{geometry}
%\usepackage{lmodern} % Ensure lmodern is loaded for pdflatex
\usepackage{tfrupee} % Include tfrupee package

\setlength{\headheight}{1cm} % Set the height of the header box
\setlength{\headsep}{0mm}  % Set the distance between the header box and the top of the text

\usepackage{gvv-book}
\usepackage{gvv}
\usepackage{cite}
\usepackage{amsmath,amssymb,amsfonts,amsthm}
\usepackage{algorithmic}
\usepackage{graphicx}
\usepackage{float}
\usepackage{textcomp}
\usepackage{xcolor}
\usepackage{txfonts}
\usepackage{listings}
\usepackage{enumitem}
\usepackage{mathtools}
\usepackage{gensymb}
\usepackage{comment}
\usepackage[breaklinks=true]{hyperref}
\usepackage{tkz-euclide} 
\usepackage{listings}
% \usepackage{gvv}                                        
\def\inputGnumericTable{}                                 
\usepackage[latin1]{inputenc}                                
\usepackage{color}                                            
\usepackage{array}                                            
\usepackage{longtable}                                       
\usepackage{calc}                                             
\usepackage{multirow}                                         
\usepackage{hhline}                                           
\usepackage{ifthen}                                           
\usepackage{lscape}
\usepackage{tikz}
\usetikzlibrary{patterns}

\begin{document}

\bibliographystyle{IEEEtran}
\vspace{3cm}

\title{5.2.65}
\author{ee25btech11063-vejith}

\maketitle
% \maketitle
% \newpage
% \bigskip
{\let\newpage\relax\maketitle}
\renewcommand{\thefigure}{\theenumi}
\renewcommand{\thetable}{\theenumi}
\setlength{\intextsep}{10pt} % Space between text and floats
\textbf{Question}\\
Solve\\
$\Vec{X}$+$\Vec{Y}$=$\begin{pmatrix}
    5 & 2\\
    0 & 9
\end{pmatrix}$ and $\Vec{X}$-$\Vec{Y}$=$\begin{pmatrix}
    3 & 6\\
    0 & -1
\end{pmatrix}$\\
\textbf{Solution}:\\
Given,
\begin{align}
    \Vec{X}+\Vec{Y}=\begin{pmatrix}
    5 & 2\\
    0 & 9
    \end{pmatrix}
\end{align}
\begin{align}
    \Vec{X}-\Vec{Y}=\begin{pmatrix}
    3 & 6\\
    0 & -1
\end{pmatrix}
\end{align}
\begin{align}
    \implies \begin{pmatrix}
        1 & 1\\
         1 & -1
    \end{pmatrix} \brak{\vec{X}\hspace{0.5cm}\vec{Y}}=\brak{\vec{A}\hspace{0.5cm}\vec{B}}\\
     \vec{A}= \begin{pmatrix}
    5 & 2\\
    0 & 9
\end{pmatrix} \text{ and } \vec{B}=\begin{pmatrix}
    3 & 6\\
    0 & -1
\end{pmatrix}
    \end{align}
Forming the Augmented matrix
\begin{align}
\left(
\begin{array}{cc|c}
1 & 1 & \vec{A} \;\; \vec{B} \\
1 & -1 &
\end{array}
\right)  &\xrightarrow{R_2 \rightarrow R_2-R_1} \left(
\begin{array}{cc|c}
1 & 1 & \vec{A} \;\; \vec{B-A} \\
0 & -2 &
\end{array} \right)\\ 
\implies -2\Vec{Y}=\vec{B-A}\\
\vec{B-A}= \begin{pmatrix}
    -2 & 4\\
    0 & -10
\end{pmatrix}\\
\implies \Vec{Y}=\begin{pmatrix} 1 & -2 \\ 0 & 5 \end{pmatrix}\\
\implies \Vec{X}+\Vec{Y}=\begin{pmatrix} 5 & 2 \\ 0 & 9 \end{pmatrix}\\
\implies \Vec{X}=\begin{pmatrix} 4 & 4 \\ 0 & 4 \end{pmatrix}
\end{align}
on back substitution we get
\begin{align}
    \implies \Vec{X}=\begin{pmatrix} 4 & 4 \\ 0 & 4 \end{pmatrix} \text{ and }
    \Vec{Y}=\begin{pmatrix} 1 & -2 \\ 0 & 5 \end{pmatrix}
\end{align}


\end{document}
