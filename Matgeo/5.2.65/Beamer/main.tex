\documentclass{beamer}
\mode<presentation>
\usepackage{amsmath,amssymb,mathtools}
\usepackage{textcomp}
\usepackage{gensymb}
\usepackage{adjustbox}
\usepackage{subcaption}
\usepackage{enumitem}
\usepackage{multicol}
\usepackage{listings}
\usepackage{url}
\usepackage{graphicx} % <-- needed for images
\def\UrlBreaks{\do\/\do-}

\usetheme{Boadilla}
\usecolortheme{lily}
\setbeamertemplate{footline}{
  \leavevmode%
  \hbox{%
  \begin{beamercolorbox}[wd=\paperwidth,ht=2ex,dp=1ex,right]{author in head/foot}%
    \insertframenumber{} / \inserttotalframenumber\hspace*{2ex}
  \end{beamercolorbox}}%
  \vskip0pt%
}
\setbeamertemplate{navigation symbols}{}

\lstset{
  frame=single,
  breaklines=true,
  columns=fullflexible,
  basicstyle=\ttfamily\tiny   % tiny font so code fits
}

\numberwithin{equation}{section}

% ---- your macros ----
\providecommand{\nCr}[2]{\,^{#1}C_{#2}}
\providecommand{\nPr}[2]{\,^{#1}P_{#2}}
\providecommand{\mbf}{\mathbf}
\providecommand{\pr}[1]{\ensuremath{\Pr\left(#1\right)}}
\providecommand{\qfunc}[1]{\ensuremath{Q\left(#1\right)}}
\providecommand{\sbrak}[1]{\ensuremath{{}\left[#1\right]}}
\providecommand{\lsbrak}[1]{\ensuremath{{}\left[#1\right.}}
\providecommand{\rsbrak}[1]{\ensuremath{\left.#1\right]}}
\providecommand{\brak}[1]{\ensuremath{\left(#1\right)}}
\providecommand{\lbrak}[1]{\ensuremath{\left(#1\right.}}
\providecommand{\rbrak}[1]{\ensuremath{\left.#1\right)}}
\providecommand{\cbrak}[1]{\ensuremath{\left\{#1\right\}}}
\providecommand{\lcbrak}[1]{\ensuremath{\left\{#1\right.}}
\providecommand{\rcbrak}[1]{\ensuremath{\left.#1\right\}}}
\theoremstyle{remark}
\newtheorem{rem}{Remark}
\newcommand{\sgn}{\mathop{\mathrm{sgn}}}
\providecommand{\abs}[1]{\left\vert#1\right\vert}
\providecommand{\res}[1]{\Res\displaylimits_{#1}}
\providecommand{\norm}[1]{\lVert#1\rVert}
\providecommand{\mtx}[1]{\mathbf{#1}}
\providecommand{\mean}[1]{E\left[ #1 \right]}
\providecommand{\fourier}{\overset{\mathcal{F}}{ \rightleftharpoons}}
\providecommand{\system}{\overset{\mathcal{H}}{ \longleftrightarrow}}
\providecommand{\dec}[2]{\ensuremath{\overset{#1}{\underset{#2}{\gtrless}}}}
\newcommand{\myvec}[1]{\ensuremath{\begin{pmatrix}#1\end{pmatrix}}}
\let\vec\mathbf

\title{Matgeo Presentation - Problem 5.2.65}
\author{ee25btech11063 - Vejith}

\begin{document}


\frame{\titlepage}
\begin{frame}{Question}
Solve\\
$\Vec{X}$+$\Vec{Y}$=$\begin{pmatrix}
    5 & 2\\
    0 & 9
\end{pmatrix}$ and $\Vec{X}$-$\Vec{Y}$=$\begin{pmatrix}
    3 & 6\\
    0 & -1
\end{pmatrix}$
\end{frame}

\begin{frame}{Solution}
Given,
\begin{align}
    \Vec{X}+\Vec{Y}=\begin{pmatrix}
    5 & 2\\
    0 & 9
    \end{pmatrix}
\end{align}
\begin{align}
    \Vec{X}-\Vec{Y}=\begin{pmatrix}
    3 & 6\\
    0 & -1
\end{pmatrix}
\end{align}
\begin{align}
    \implies \begin{pmatrix}
        1 & 1\\
         1 & -1
    \end{pmatrix} \myvec{\vec{X}\\\vec{Y}}=\myvec{\vec{A}\\ \vec{B}}\\
     \vec{A}= \begin{pmatrix}
    5 & 2\\
    0 & 9
\end{pmatrix} \text{ and } \vec{B}=\begin{pmatrix}
    3 & 6\\
    0 & -1
\end{pmatrix}
    \end{align}
Forming the Augmented matrix
\begin{align}
\left(\begin{array}{cc|c}
1 & 1 & \vec{A}\\
1 & -1 & \vec{B}
\end{array}\right)  &\xrightarrow{R_2 \rightarrow R_2-R_1} \left(\begin{array}{cc|c}
1 & 1 & \vec{A}\\
0 & -2 & \vec{B-A}
\end{array}\right)\\
\implies -2\Vec{Y}=\vec{B-A}
\end{align}
\end{frame}

\begin{frame}{Solution}
    \begin{align}
\vec{B-A}= \begin{pmatrix}
    -2 & 4\\
    0 & -10
\end{pmatrix}\\
\implies \Vec{Y}=\begin{pmatrix} 1 & -2 \\ 0 & 5 \end{pmatrix}\\
\implies \Vec{X}+\Vec{Y}=\begin{pmatrix} 5 & 2 \\ 0 & 9 \end{pmatrix}\\
\implies \Vec{X}=\begin{pmatrix} 4 & 4 \\ 0 & 4 \end{pmatrix}
\end{align}
on back substitution we get
    \begin{align}
    \implies \Vec{X}=\begin{pmatrix}
    4 & 4\\
    0 & 4
\end{pmatrix} \text{ and } \Vec{Y}=\begin{pmatrix}
    1 & -2\\
    0 & 5
\end{pmatrix}
\end{align}
\end{frame}

% --------- CODE APPENDIX ---------
\section*{Appendix: Code}

% C program
\begin{frame}[fragile]{C Code: matrix.c}
\begin{lstlisting}[language=C]
#include <stdio.h>

int main() {
    FILE *fp;
    fp = fopen("matrix.dat", "w");  // open file for writing
    if (fp == NULL) {
        printf("Error opening file!\n");
        return 1;
    }

    // Define the given matrices
    int A[2][2] = { {5, 2}, {0, 9} };  // X + Y
    int B[2][2] = { {3, 6}, {0, -1} }; // X - Y
    int X[2][2], Y[2][2];

    // Calculate X = (A + B)/2
    for (int i = 0; i < 2; i++) {
        for (int j = 0; j < 2; j++) {
            X[i][j] = (A[i][j] + B[i][j]) / 2;
        }
    }

    // Calculate Y = (A - B)/2
    for (int i = 0; i < 2; i++) {
        for (int j = 0; j < 2; j++) {
            Y[i][j] = (A[i][j] - B[i][j]) / 2;
        }
    }

    // Write results to file
    fprintf(fp, "Matrix X:\n");
    for (int i = 0; i < 2; i++) {
        for (int j = 0; j < 2; j++) {
           
\end{lstlisting}
\end{frame}

\begin{frame}[fragile]{C Code: matrix.c}
\begin{lstlisting}[language=C]
 fprintf(fp, "%d ", X[i][j]);
        }
        fprintf(fp, "\n");
    }

    fprintf(fp, "\nMatrix Y:\n");
    for (int i = 0; i < 2; i++) {
        for (int j = 0; j < 2; j++) {
            fprintf(fp, "%d ", Y[i][j]);
        }
        fprintf(fp, "\n");
    }

    fclose(fp);
    printf("Solution written to matrix.dat\n");
    return 0;
}
\end{lstlisting}
\end{frame}

\begin{frame}[fragile]{Python: solution.py}
\begin{lstlisting}[language=Python]
 import numpy as np

# Given matrices
A = np.array([[5, 2],
              [0, 9]])   # X + Y
B = np.array([[3, 6],
              [0, -1]])  # X - Y

# Solve for X and Y
X = (A + B) / 2
Y = (A - B) / 2

print("Matrix X:")
print(X)
print("\nMatrix Y:")
print(Y)



\end{lstlisting}
\end{frame} 
\end{document}
