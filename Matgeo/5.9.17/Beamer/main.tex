\documentclass{beamer}
\mode<presentation>
\usepackage{amsmath,amssymb,mathtools}
\usepackage{textcomp}
\usepackage{gensymb}
\usepackage{adjustbox}
\usepackage{subcaption}
\usepackage{enumitem}
\usepackage{multicol}
\usepackage{listings}
\usepackage{url}
\usepackage{graphicx} % <-- needed for images
\def\UrlBreaks{\do\/\do-}

\usetheme{Boadilla}
\usecolortheme{lily}
\setbeamertemplate{footline}{
  \leavevmode%
  \hbox{%
  \begin{beamercolorbox}[wd=\paperwidth,ht=2ex,dp=1ex,right]{author in head/foot}%
    \insertframenumber{} / \inserttotalframenumber\hspace*{2ex}
  \end{beamercolorbox}}%
  \vskip0pt%
}
\setbeamertemplate{navigation symbols}{}

\lstset{
  frame=single,
  breaklines=true,
  columns=fullflexible,
  basicstyle=\ttfamily\tiny   % tiny font so code fits
}

\numberwithin{equation}{section}

% ---- your macros ----
\providecommand{\nCr}[2]{\,^{#1}C_{#2}}
\providecommand{\nPr}[2]{\,^{#1}P_{#2}}
\providecommand{\mbf}{\mathbf}
\providecommand{\pr}[1]{\ensuremath{\Pr\left(#1\right)}}
\providecommand{\qfunc}[1]{\ensuremath{Q\left(#1\right)}}
\providecommand{\sbrak}[1]{\ensuremath{{}\left[#1\right]}}
\providecommand{\lsbrak}[1]{\ensuremath{{}\left[#1\right.}}
\providecommand{\rsbrak}[1]{\ensuremath{\left.#1\right]}}
\providecommand{\brak}[1]{\ensuremath{\left(#1\right)}}
\providecommand{\lbrak}[1]{\ensuremath{\left(#1\right.}}
\providecommand{\rbrak}[1]{\ensuremath{\left.#1\right)}}
\providecommand{\cbrak}[1]{\ensuremath{\left\{#1\right\}}}
\providecommand{\lcbrak}[1]{\ensuremath{\left\{#1\right.}}
\providecommand{\rcbrak}[1]{\ensuremath{\left.#1\right\}}}
\theoremstyle{remark}
\newtheorem{rem}{Remark}
\newcommand{\sgn}{\mathop{\mathrm{sgn}}}
\providecommand{\abs}[1]{\left\vert#1\right\vert}
\providecommand{\res}[1]{\Res\displaylimits_{#1}}
\providecommand{\norm}[1]{\lVert#1\rVert}
\providecommand{\mtx}[1]{\mathbf{#1}}
\providecommand{\mean}[1]{E\left[ #1 \right]}
\providecommand{\fourier}{\overset{\mathcal{F}}{ \rightleftharpoons}}
\providecommand{\system}{\overset{\mathcal{H}}{ \longleftrightarrow}}
\providecommand{\dec}[2]{\ensuremath{\overset{#1}{\underset{#2}{\gtrless}}}}
\newcommand{\myvec}[1]{\ensuremath{\begin{pmatrix}#1\end{pmatrix}}}
\let\vec\mathbf

\title{Matgeo Presentation - 5.9.17}
\author{ee25btech11063 - Vejith}

\begin{document}


\frame{\titlepage}
\begin{frame}{Question}
If   $\begin{pmatrix}
    2a+b & a-2b\\
    5c-d & 4c+3d
\end{pmatrix}$=$\begin{pmatrix}
    4 & -3\\
    11 & 24
\end{pmatrix}$, then the value of a+b-c+2d
\end{frame}

 \begin{frame}{Solution}
     From the matrix equation the first row gives
\begin{align}
    \begin{pmatrix}
        2 & 1 & 0 & 0\\
        1 & -2 & 0 & 0
        \end{pmatrix} \myvec{a\\b\\c\\d}=\myvec{4\\-3}
    \end{align}
    From the matrix equation the second row gives
    \begin{align}
    \begin{pmatrix}
        0 & 0 & 5 & -1\\
        0 & 0 & 4 & 3
        \end{pmatrix} \myvec{a\\b\\c\\d}=\myvec{11\\24}
    \end{align}
combine (0.1) and (0.2)
\begin{align}
    \begin{pmatrix}
        2 & 1 & 0 & 0\\
        1 & -2 & 0 & 0\\
        0 & 0 & 5 & -1\\
        0 & 0 & 4 & 3
        \end{pmatrix} \myvec{a\\b\\c\\d}=\myvec{4\\-3\\11\\24}
\end{align}
\end{frame}

\begin{frame}{Solution}
Forming the augmented matrix\\
\begin{align}
    \left(\begin{array}{cccc|c}
       2 & 1 & 0 & 0 & 4\\
        1 & -2 & 0 & 0 & -3\\
        0 & 0 & 5 & -1 & 11\\
        0 & 0 & 4 & 3 & 24
\end{array}\right) &\xrightarrow{R_2 \leftrightarrow R_2-\frac{1}{2}\times R_1}
\left(\begin{array}{cccc|c}
       2 & 1 & 0 & 0 & 4\\
        0 & -\frac{5}{2} & 0 & 0 & -5\\
        0 & 0 & 5 & -1 & 11\\
        0 & 0 & 4 & 3 & 24
        \end{array}\right)\\ &\xrightarrow{R_1 \leftrightarrow R_1+\frac{2}{5}\times R_2} \left(\begin{array}{cccc|c}
       2 & 0 & 0 & 0 & 2\\
        0 & -\frac{5}{2} & 0 & 0 & -5\\
        0 & 0 & 5 & -1 & 11\\
        0 & 0 & 4 & 3 & 24
        \end{array}\right)\\ &\xrightarrow{R_4 \leftrightarrow R_4-\frac{4}{5}\times R_3} \left(\begin{array}{cccc|c}
       2 & 0 & 0 & 0 & 2\\
        0 & -\frac{5}{2} & 0 & 0 & -5\\
        0 & 0 & 5 & -1 & 11\\
        0 & 0 & 0 & \frac{19}{5} & \frac{76}{5}
        \end{array}\right)
        \end{align}
        \end{frame}
        \begin{frame}{Solution}
            \begin{align}
         &\xrightarrow{R_3 \leftrightarrow R_3+\frac{5}{19}\times R_4} \left(\begin{array}{cccc|c}
       2 & 0 & 0 & 0 & 2\\
        0 & -\frac{5}{2} & 0 & 0 & -5\\
        0 & 0 & 5 & 0 & 15\\
        0 & 0 & 0 & \frac{19}{5} & \frac{76}{5}
        \end{array}\right)
        \end{align}
        on back substitution we get
        \begin{align}
        \implies \myvec{a\\b\\c\\d}=\myvec{1\\2\\3\\4}
\end{align}
\end{frame}

\begin{frame}{Conclusion}
    value of 
    \begin{align}
    a+b-c+2d=    \brak{1\hspace{0.3cm}1\hspace{0.3cm}-1\hspace{0.3cm}2}\myvec{a\\b\\c\\d}\\=\brak{1\hspace{0.3cm}1\hspace{0.3cm}-1\hspace{0.3cm}2}\myvec{1\\2\\3\\4}=8
    \end{align}
    \end{frame}


 % --------- CODE APPENDIX ---------
\section*{Appendix: Code}

% C program
\begin{frame}[fragile]{C Code: code.c}
\begin{lstlisting}[language=C]
#include <stdio.h>

void solve2x2(double aug[2][3], int *x, int *y) {
    // Gaussian elimination for 2x2 system
    // aug is 2x3 augmented matrix

    // Step 1: make leading coefficient of first row = 1
    if (aug[0][0] != 0) {
        double factor = aug[0][0];
        aug[0][0] /= factor;
        aug[0][1] /= factor;
        aug[0][2] /= factor;
    }

    // Step 2: eliminate below
    double factor = aug[1][0];
    aug[1][0] -= factor * aug[0][0];
    aug[1][1] -= factor * aug[0][1];
    aug[1][2] -= factor * aug[0][2];

    // Step 3: make pivot in row 2 equal to 1
    if (aug[1][1] != 0) {
        factor = aug[1][1];
        aug[1][0] /= factor;
        aug[1][1] /= factor;
        aug[1][2] /= factor;
    }

    // Step 4: eliminate above row 2
    factor = aug[0][1];
    aug[0][0] -= factor * aug[1][0];
    aug[0][1] -= factor * aug[1][1];
    aug[0][2] -= factor * aug[1][2];
\end{lstlisting}
\end{frame}

\begin{frame}[fragile]{C Code: code.c}
\begin{lstlisting}[language=C]
    // Now solution is in aug[0][2], aug[1][2]
    *x = (int)(aug[0][2] + 0.5); // rounding
    *y = (int)(aug[1][2] + 0.5);
}

int main() {
    int a, b, c, d, result;

    double aug1[2][3] = {
        {2,  1,  4},
        {1, -2, -3}
    };
    solve2x2(aug1, &a, &b);

    
    double aug2[2][3] = {
        {5, -1, 11},
        {4,  3, 24}
    };
    solve2x2(aug2, &c, &d);

    
    result = a + b - c + 2 * d;

  
    FILE *fp = fopen("answer.dat", "w");
    if (fp == NULL) {
        printf("Error opening file!\n");
        return 1;
    }
    fprintf(fp, "%d\n", result);
    fclose(fp);
    return 0;}
\end{lstlisting}
\end{frame}

\begin{frame}[fragile]{Python: solution.py}
\begin{lstlisting}[language=Python]
import numpy as np

# -----------------------------
# System 1: for a, b
# 2a + b  =  4
#  a - 2b = -3
# -----------------------------
A1 = np.array([[2, 1],
               [1, -2]])
B1 = np.array([4, -3])

# Solve for a, b
a, b = np.linalg.solve(A1, B1)

# -----------------------------
# System 2: for c, d
# 5c - d = 11
# 4c + 3d = 24
# -----------------------------
A2 = np.array([[5, -1],
               [4,  3]])
B2 = np.array([11, 24])

# Solve for c, d
c, d = np.linalg.solve(A2, B2)
# -----------------------------
# Compute expression
# -----------------------------
result = a + b - c + 2*d
print("a =", a, " b =", b)
print("c =", c, " d =", d)
print("a + b - c + 2d =", result)

\end{lstlisting}
\end{frame} 
\end{document}
